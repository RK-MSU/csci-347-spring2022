\documentclass[11pt]{article}

% set these commands
\newcommand{\course}{CSCI 347}
\newcommand{\proj}{Homework 04}

\usepackage{macros}


\begin{document}

{ ~\\
    \course \\ 
    \proj \\ 
}

Show your work. Include any code snippets you used to generate an answer, using
comments in the code to clearly indicate which problem corresponds to which code

\begin{enumerate}

    \item (2 points) Consider matrix $A$ and vector $v$.
    Compute the matrix-vector product $Av$.
    $$
        A = \begin{pmatrix} 2 & 1 \\ 1 & 3 \end{pmatrix},
        v = \begin{pmatrix}-1 & 1\end{pmatrix}
    $$

    \item Consider matrix $A$ and data set $D$:
    $$
        A = \begin{pmatrix}
            \frac{\sqrt{3}}{2} & -\frac{1}{2} \\ 
            \frac{1}{2} & \frac{\sqrt{3}}{2}
        \end{pmatrix},
        D = \begin{pmatrix}
            1  &  1.5 \\
            1  &  2 \\
            3  &  4 \\
            -1 &  -1 \\
            -1 &  1 \\
            1  & -2 \\
            2  &  2 \\
            2  & 3
        \end{pmatrix}
    $$

    \begin{enumerate}

        \item (2 points) Let $X_1$ and $X_2$ be the first and second attributes
        of the data, respectively.  Use Python to create a scatter plot of the
        data, where the $x$-axis is $X_1$ and the $y$-axis is~$X_2$.

        \item (4 points) Treating each row as a 2-dimensional vector, apply the
        linear transformation $A$ to each row.  In other words, let $x_i$ be the
        $i$-th row of $D$. For each $x_i$, find the matrix-vector product $A
        x_i$.  For example, $x_2 = \begin{pmatrix} 1 \\  2 \end{pmatrix}$.

        \item (3 points) Use Python to create a plot showing both the original
        data and the transformed data, with the $x$-axis still corresponding to
        $X_1$ and the y-axis corresponding to $X_2$. Use different colors and
        markers to differentiate between the original and transformed data. That
        is, each transformed data point in the plot should be one matrix-vector
        product $Ax_i$, which is a 2-dimensional vector. Each original point in
        the plot should have the same coordinates as it did in part 2.1.

        \item (1 point) Write down the multi-dimensional mean of the data.
        (Remember that this should be a 2-dimensional vector)

        \item (2 points) Mean-center the data. Write down the mean-centered data
        matrix.

        \item (2 points] Use Python to create a scatter plot showing both the
        original data and the mean-centered data, where the $x$-axis is $X_1$
        and the $y$-axis is $X_2$. Use different colors and markers to
        differentiate between the original and mean-centered data.

        \item (3 points) Write down the covariance matrix of the data matrix
        $D$. Use estimated covariance.

        \item (3 points) Write down the covariance matrix of the centered data
        matrix $Z$. Use estimated covariance.

        \item (3 points) Write down the covariance matrix of the data after
        applying standard normalization.

    \end{enumerate}

\end{enumerate}

{\bf Acknowledgements:} Homework problems adapted from assignments of
Veronika Strnadova-Neeley.

\end{document}
