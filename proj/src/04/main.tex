\documentclass[11pt]{article}
\usepackage{hyperref}

% set these commands
\newcommand{\course}{CSCI 347}
\newcommand{\proj}{Project 04: Final Project}

\usepackage{macros}


\begin{document}

{ ~\\
    \course \\ 
    \proj \\ 
}

This project may be completed individually or a group of up three people. You
may ask others for help, but the submitted work must be from the group. You may
use online resources, but they must be cited.

This project is much more open-ended than previous projects. You are encouraged
to explore a data mining topic of interest. You may choose to dive deeper into a
topic covered in class (for example, improvements/extensions of $k$-means), or
explore a related topic that we did not have time to cover (for example,
additional clustering or classification algorithms, advanced feature selection
algorithms, non-linear dimensionality reduction algorithms, etc.). The learning
objectives of this project are to:

\begin{itemize}

    \item Identify problems that can be solved or partially solved using data
    mining techniques.

    \item Apply appropriate data mining algorithms to a real-world data set
    using the Python programming language.

    \item Construct an end-to-end computational pipeline to solve a data mining
    problem.

    \item Explore a data mining application of interest

\end{itemize}

Keep in mind that we have limited time for this project. Some exploration may
therefore need to be left for future work.

\section*{Part 1: Plan (20 points)}

{\bf NOTE THAT PART 1 IS DUE EARLIER THAN THE REST OF THE PROJECT.}
\linebreak

Find a problem that you are interested in that has an associated data set. You
can browse the UCI Machine Learning Repository, the SNAP collection, Kaggle, or
any other source of publicly available data. Think about how you might apply
data mining to this problem. Write one paragraph that:

\begin{itemize}

    \item Summarizes the problem

    \item Summarizes the data set.  For example, include how many instances and
    attributes, how many categorical and numerical features, how many nodes and
    edges if using graph data, etc.

    \item Lists the data mining techniques you would like to use to help solve
    this problem.

    \item Describes what part of your proposed solution may need to be left for
    future work if you run out of time.

\end{itemize}

The paragraph summarizing your proposed work must be turned in by the Part 1 due
date. You are encouraged to visit the instructor or TA office hours to help
develop your idea.


\section*{Part 2: Implement (30 points)}

Write code to analyze your data. This should include pre-processing such as
missing value imputation and one-hot encoding, dimensionality reduction, and any
data mining algorithms that you want to apply to your data.

\section*{Part 3: Report (40 points)}

Write up a report summarizing your findings. Summarize the methods you applied,
from beginning to end, including pre-processing techniques, dimensionality
reduction, clustering or classification, etc. Include answers to the following
questions in your report:

\begin{itemize}
    \item What problem were you trying to solve or help solve?

    \item Describe the data:
    \begin{itemize}
        \item How many instances?
        \item How many attributes?
        \item Any missing values?
        \item Number of categorical and numeric attributes?
    \end{itemize}

    \item What pre-processing techniques did you apply and why? Make sure to
    justify the use of each technique you used. For example label encoding
    vs.~one-hot encoding.

    \item What data mining techniques did you apply and why? Make sure to
    justify the use of each technique you used. For example, why did you use
    $k$-means instead of DBSCAN.

    \item Include relevant visualizations and tables summarizing your data and
    your findings. This may include:
    \begin{itemize}

        \item a table listing the number attributes, missing values, number of
        classes, parameter settings, etc.

        \item visualization of a large graph if you are working with graph data.

        \item one or more visualizations of your data in two dimensions
        (original dimensions or PCA dimensions).

        \item for PCA, a plot of r vs. f(r).

        \item for $k$-means, a plot of the objective function for various $k$'s.

        \item for DBSCAN, a plot or table of the precision at various
        parameters.

        \item other visualizations or tables that you think will effectively
        communicate your ideas.

    \end{itemize}

    \item What did you learn through your analysis?

    \item Was anything about your results surprising or unexpected?

    \item How will your work help with understanding the problem you set out to
    solve?

    \item What else would you do if you had more time?
\end{itemize}

\section*{Part 4: Present (10 points)}

Make a 5-10 minute video presentation summarizing your findings. You may use
whatever video editing technology you prefer. (The MSU supported tool is
TechSmith Relay.  See the
\href{https://ato.montana.edu/accessforall/tutorials/techsmith/5-steps-to-record-techsmith-video.html}{UIT
tutorial} for more info.) The video should:

\begin{itemize}

    \item State your name.

    \item Summarize your project, including:
    \begin{itemize}

        \item the problem you are interested in.

        \item what data mining techniques you used to analyze data related to
        the problem.

    \end{itemize}

    \item Your key findings and any surprising results.

    \item What else you would work on if you had more time.

\end{itemize}

The goal is to summarize the work you have done and what you have learned from
the process.
\linebreak

\noindent
{\bf Note: any presentation that exceeds 10 minutes or does not reach 5
minutes will be docked 1 point per minute.}

\section*{Tips and Acknowledgements}

Make sure to submit your code, report, and video on Brightspace. The report
should also be turned in on Gradescope.

{\bf Acknowledgements:} Project adapted from assignments of Veronika
Strnadova-Neeley.

\end{document}
