\documentclass[11pt]{article}
\usepackage{hyperref}

% set these commands
\newcommand{\course}{CSCI 347}
\newcommand{\proj}{Project 01: Exploratory Data Analysis}

\usepackage{macros}


\begin{document}

{ ~\\
    \course \\ 
    \proj \\ 
}

Partner work is allowed on this project. Browse through the
\href{https://archive.ics.uci.edu/ml/index.php}{UCI Machine Learning Repository}
to find a data set that is interesting to you, and has both categorical and
numerical data (but is small enough to work with). \textit{Note}: If your chosen
data set is not small enough to download, take a sample of the data set that
contains at least 200 entities (instances/rows).

\section*{Problem 1: Write the Introduction}

In a well-written paragraph, answer the following questions about the data:

\begin{itemize}

    \item (4 points) What was the data used for?

    \item (2 points) Who (or what organization) uploaded the data?

    \item (5 points) How many attributes and how many entities are represented
    in the data?

    \begin{itemize}
        \item How many numerical attributes?
        \item How many categorical attributes?
        \item Would you suggest that each categorical attribute be label-encoded
        or one-hot-encoded?
        \item Why did you choose the encoding?
    \end{itemize}

    \item (4 points) Are there missing values in the data? If so, what
    proportion of the data is missing overall? What proportion of data is
    missing per attribute (you may use a plot or table to summarize this
    information)?

    \item (7 points) Why is this data set interesting to you?

    \item (6 points) Of the attributes used to describe this data, which do you
    think are the most descriptive of the data and why (before doing any data
    analysis)?

\end{itemize}



\section*{Part 2: Write Python code for data analysis}

Use Python to write the following functions, without using any functions with
the same purpose in sklearn, pandas, numpy, or any other library (though you may
want to use these libraries to check your answers):

\begin{itemize}

    \item (5 points) A function that will compute the mean of a numerical,
    multidimensional data set input as a 2-dimensional numpy array

    \item (5 points) A function that will compute the estimated covariance
    between two attributes that are input as one-dimensional numpy vectors

    \item (5 points) A function that will compute the correlation between two
    attributes that are input as two numpy vectors.

    \item (5 points) A function that will normalize the attributes in a
    two-dimensional numpy array using range normalization.

    \item (5 points) A function that will normalize the attributes in a
    two-dimensional numpy array using standard normalization.

    \item (5 points) A function that will compute the covariance matrix of a
    data set.

    \item (5 points) A function that will label-encode a two-dimensional
    categorical data array that is passed in as input.

\end{itemize}

\section*{Part 3: Analyze the data with your code and write up the results}

Use your code from Part 2 to answer the following questions in a well-written
paragraph, and create the following plots from the numerical portion of the
data. Use your functions to compute the multi-variate mean and covariance matrix
of the \textbf{numerical portion} of your data set. \textbf{Before answering the
questions}:

\begin{itemize}

    \item (5 points) Convert all categorical attributes using label encoding or
    one-hot-encoding

    \item (2 points) If your data has missing values, fill in those values with
    the attribute mean.

\end{itemize}

\noindent Questions to answer:
\begin{itemize}

    \item (2 points) What is the multivariate mean of the numerical data matrix
    (where categorical data have been converted to numerical values)?

    \item (4 points) What is the covariance matrix of the numerical data matrix
    (where categorical data have been converted to numerical values)?

    \item (5 points) Choose 5 pairs of attributes that you think could be
    related. Create scatter plots of all 5 pairs and include these in your
    report, along with a description and analysis that summarizes why these
    pairs of attributes might be related, and how the scatter plots do or do not
    support this intuition.

    \item (3 points) Which range-normalized numerical attributes have the
    greatest estimated covariance? What is their estimated covariance? Create a
    scatter plot of these range-normalized attributes.

    \item (3 points) Which Z-score-normalized numerical attributes have the
    greatest correlation? What is their correlation? Create a scatter plot of
    these Z-score-normalized attributes.

    \item (3 points) Which Z-score-normalized numerical attributes have the
    smallest correlation? What is their correlation? Create a scatter plot of
    these Z-score-normalized attributes.

    \item (3 points) How many pairs of features have correlation greater than or
    equal to 0.5?

    \item (3 points) How many pairs of features have negative estimated
    covariance?

    \item (2 points) What is the total variance of the data?

    \item (2 points) What is the total variance of the data, restricted to the
    five features that have the greatest estimated variance?

\end{itemize}

\section*{Tips and Acknowledgements}

Make sure to submit your answer as a PDF on Gradscope and Brightspace. Make sure
to show your work. Include any code snippets you used to generate an answer,
using comments in the code to clearly indicate which problem corresponds to
which code.

{\bf Acknowledgements:} Project adapted from assignments of Veronika
Strnadova-Neeley.

\end{document}
